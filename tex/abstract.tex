Ant foraging behavior is a collective decision making process in which, through  pheromone deposition and individual interactions between ants, a colony of ants selects and exploits a path between their nest and a food source.
Research into the collective decision making strategies of ants, in addition to characterizing the biological mechanisms and emergent properties of the foraging process, has the potential to be leveraged into applications such as swarm robotics and commercial logistics management.
Although ant foraging behavior has been extensively studied on flat terrains, ant foraging over uneven terrains is not well studied \cite{Clune2008HowDecreases}.
This research presents an individual-based set of differential equations to model ant foraging behavior over uneven terrain in an enclosed arena. This model is employed to investigate the characteristics of foraging paths that ants tend towards when foraging over simple inclines of varying magnitudes.
Numerical solutions of the model predict that, over most inclines, ants tend to favor the direct path between nest and food, with the direct path typically being more strongly favored when foraging over steep inclines \cite{Reisinger2007AcquiringRepresentations}.